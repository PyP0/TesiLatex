% ************************** Thesis Abstract *****************************
% Use `abstract' as an option in the document class to print only the titlepage and the abstract.
\begin{abstract}
	
In questi ultimi anni i droni sono diventati un argomento di tendenza, specialmente in ambito civile e della ricerca, ed emergono continuamente nuove opportunità per impiegarli. \\
Uno dei settori in cui stanno acquisendo maggior importanza è quello che li vede impiegati come unità di supporto alle squadre di primo soccorso, in seguito al verificarsi di calamità naturali. 
Oltre ad utilizzarli per individuare i dispersi o ispezionare infrastrutture pericolanti, i droni possono venire equipaggiati con Access Points wireless e dislocati nell'area per configurare una recovery ad-hoc flying network, con lo scopo di instaurare velocemente un sistema di comunicazione d'emergenza e interconnettere gli utenti (come civili o membri delle squadre di soccorso) presenti nell'area di emergenza. Ciascun utente nel range di un drone potrà quindi comunicare con esso e, tramite multi-hop relaying, con tutti gli altri nodi interconnessi alla rete. \\
Da questa tipologia di impiego, in continuo sviluppo, emerge un interessante problema di ottimizzazione, che riguarda la determinazione della posizione del minor numero possibile di droni per garantire che tutti i canali di comunicazione tra gli utenti richiesti vengano stabiliti. \\
L'impiego di canali wireless pone però il problema dell'interferenza radio sulle comunicazioni, e di conseguenza proponiamo un modello di interferenza che viene integrato nel modello di Mixed Integer Linear Programming chiamato I-DARNC (Interference-aware Drone Ad-hoc Relay Network Configuration problem), da noi sviluppato. \\
Poiché la dimensione del modello tende a crescere molto velocemente con il numero degli utenti, proponiamo inoltre una procedura euristica per affrontare anche le istanze più complesse del dataset preso in esame, e successivamente presentiamo i risultati ottenuti.
%Nowadays drones have become a trending topic, especially in civil and public fields, and always new ways of use are being researched. \\ They are becoming more and more important in emergency response scenarios, where they can be equipped with wireless access points and deployed to configure a recovery ad-hoc flying network, e.g., to interconnect users (like civilians or emergency response teams) positioned in a disaster area. Communication between users is allowed by means of multi-hop paths including one of more relay drones. An interesting optimization problem asks for determining the position of a minimum number of drones in order to guarantee that all the required communication paths can be established. Wireless channels are used and interference plays an important role: we propose an interference model and we embed it in a Mixed Integer Linear Programming formulation called I-DARNC (Interference-aware Drone Ad-hoc Relay Network Configuration problem). As the model tends to grow very fast with the number of users, an heuristic procedure is proposed to cope with larger instances. Preliminary computational results are presented.

\end{abstract}

