%*******************************************************************************
%*********************************** First Chapter *****************************
%*******************************************************************************

\chapter{Introduzione}  %Title of the First Chapter

\ifpdf
    \graphicspath{{Chapter1/Figs/Raster/}{Chapter1/Figs/PDF/}{Chapter1/Figs/}}
\else
    \graphicspath{{Chapter1/Figs/Vector/}{Chapter1/Figs/}}
\fi

Negli ultimi anni i droni hanno ottenuto una grande attenzione, sia in campo militare che civile, grazie soprattutto ai continui progressi tecnologici nella miniaturizzazione delle componenti (elettroniche e non) e alla progressiva riduzione dei costi delle stesse. \\
Se gli ambiti d'uso in campo militare sono più ristretti e prevedibili (sorveglianza, ricognizione, bombardamento, collegamenti radio), nel campo civile e della ricerca vengono studiati sempre nuovi modi d'impiego. 
Per esempio, si sta sperimentando l'uso dei droni in campo agricolo per monitorare il raccolto, in campo civile per ispezionare parti di edifici o altre infrastrutture difficilmente raggiungibili, nella raccolta di dati ambientali in zone pericolose per l'uomo o nel trasporto urgente di materiali.\\
Secondo un recente studio di settore della PwC (una delle compagnie di consulenza più importanti a livello mondiale), si stima che nei prossimi anni il mercato di servizi business basati sull'impiego di droni sarà valutato per oltre 127 miliardi di dollari, principalmente nei settori di infrastruttura (45.2 miliardi) e agricoltura (32.4 miliardi) \cite{pwc2016}. La popolarità dei droni sta infatti spingendo sempre più aziende a testarne l'uso per ridurre i costi di manodopera e offrire nuovi servizi. Uno dei ruoli chiave dei droni, in ambito business, è quello di data service, ovvero l'impiego come strumento per raccogliere dati e effettuare mapping di aree. \\
Un altro ambito di utilità pubblica in fase di studio riguarda l'impiego di droni nel supporto alle squadre di emergenza per la ricerca di superstiti o per stabilire un primo sistema di comunicazione in seguito a disastri ambientali che hanno danneggiato l'infrastruttura di comunicazione tradizionale. \\
In quest'ultimo caso, infatti, i droni potrebbero venir equipaggiati con dispositivi per la comunicazione Wi-Fi e posizionati, anche in maniera autonoma, per formare la backbone di una rete ad-hoc wireless che consenta comunicazioni dati o voce tra gruppi di client rimasti isolati a terra. \\
In questi scenari emerge un interessante e nuovo problema di ottimizzazione, riguardante la configurazione di una rete ad-hoc in cui i droni svolgono il ruolo di relays.
In questa tesi proporremo una possibile soluzione a questo problema, sviluppando un modello di programmazione lineare intera mista che determini il posizionamento ottimale di un gruppo di droni, visti come nodi di una rete ad-hoc, in modo da mettere in comunicazione tra loro gruppi sparsi di utenti rimasti isolati. A questa formulazione integreremo poi un modello statico per stimare l'interferenza radio causata dalle trasmissioni dei nodi. \\
Nel \chaptername\ \ref{chap:tecnologico} forniremo una visione generale d'insieme sui droni, descrivendone la storia, la struttura e le principali tipologie presenti sul mercato, mostrando come differenze apparentemente poco significative come la dimensione, il peso, il tipo di propulsione o alimentazione in realtà impongano forti vincoli sulle funzionalità e gli ambiti d'uso dello stesso. Evidenzieremo inoltre la principale problematica  dell'autonomia di volo, e alcuni casi studio che hanno cercato di risolverla.
Successivamente metteremo a confronto le reti ad-hoc classiche con le reti di veicoli e di droni, evidenziandone le differenze e le principali problematiche, dal punto di vista dei protocolli di routing, della variazione della topologia di rete e dei vincoli energetici.\\
Nel \chaptername\ \ref{chap:formulazione} formalizzeremo il problema I-DARNC (Interference-aware Drone Ad-hoc Relay Network Configuration problem) e lo modelleremo su un grafo di rete, in cui i nodi sono l'insieme di utenti e droni, e gli archi sono i collegamenti stabiliti tra di loro. Descriveremo inoltre le principali assunzioni che ci permetteranno successivamente di formulare efficacemente il problema tramite modelli di programmazione lineare intera mista. \\
Nel \chaptername\ \ref{chap:strumenti} faremo un'analisi dello stato dell'arte sull'impiego dei droni dal punto di vista del networking, mostrando gli studi che hanno affrontato un problema simile al nostro e evidenziandone le principali differenze. Illustreremo anche come le principali problematiche, descritte nel \chaptername\ \ref{chap:tecnologico}, sono state affrontate dalla ricerca. 
In seguito analizzeremo il problema dell'interferenza radio e della difficoltà intrinseca nella sua misurazione e previsione, e proporremo il nostro modello di interferenza basato sullo Shadowing Model del software ns-2, ovvero un noto modello empirico che determina l'attenuazione che un segnale radio subisce a causa di fattori ambientali e ostacoli incontrati. \\
Nel \chaptername\ \ref{chap:modello} presenteremo il nuovo modello MILP per I-DARNC, descrivendone i vincoli, e le formulazioni alternative più rilevanti emerse durante la sua progettazione. \\
Nel \chaptername\ \ref{cap:metodi} descriveremo i metodi impiegati per risolvere le istanze del nostro dataset, ovvero l'ottimizzatore CPLEX e le euristiche. 
Descriveremo la configurazione di CPLEX adottata, le motivazioni che hanno portato ad essa e le difficoltà incontrate dal solver nel risolvere istanze di dimensioni medio-grandi. 
Come conseguenza proporremo delle euristiche costruttive, sviluppate per poter risolvere le istanze più complesse, basate sulla risoluzione iterativa di versioni semplificate del modello, ottenute fissando i valori di alcune variabili secondo un criterio che, di iterazione in iterazione, tende a determinare soluzioni con valori migliorativi della funzione obiettivo. \\
Nel \chaptername\ \ref{chap:risultati} descriveremo la metodologia per la definizione di un dataset utilizzato per gli esperimenti numerici, e presenteremo i risultati ottenuti risolvendo le istanze con CPLEX e con le euristiche, confrontandoli in base alla precisione della soluzione e al tempo necessario per ottenerla.\\
Infine nel \chaptername\ \ref{chap:conclusioni} faremo un riepilogo delle tematiche affrontate in questa tesi, dei risultati ottenuti e suggeriremo le migliorie che potranno essere portate a compimento nei possibili sviluppi futuri del lavoro. \\

