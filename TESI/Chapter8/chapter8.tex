 \chapter{Conclusioni} \label{chap:conclusioni}

% **************************** Define Graphics Path **************************
\ifpdf
    \graphicspath{{Chapter8/Figs/Raster/}{Chapter8/Figs/PDF/}{Chapter8/Figs/}}
\else
    \graphicspath{{Chapter8/Figs/Vector/}{Chapter8/Figs/}}
\fi

L'impiego dei droni come strumenti di supporto per scopi civili è in costante crescita, e vengono studiati sempre nuovi metodi di impiego, spinti anche dalle concrete possibilità di business emergenti. \\
L'ambito sperimentale su cui si è concentrata questa tesi è quello di utilizzarli per il deployment rapido e autonomo di una rete di comunicazione di emergenza, da utilizzare in seguito del verificarsi di disastri naturali che hanno compromesso le reti di comunicazione tradizionali.
Per fare ciò i droni vengono equipaggiati con Access Points wireless e posizionati nel range dei client per creare una backbone FANET (Flying Ad-hoc Network), consentendo loro di connettersi e comunicare. \\
In questa tesi abbiamo affrontato lo scenario appena descritto formalizzandolo come un nuovo problema di ottimizzazione combinatoria, l'Interference-aware Drone Ad-hoc Relay Network Configuration problem (I-DARNC). Per il problema, abbiamo proposto una formulazione di programmazione lineare intera mista (Mixed Integer Linear Programming - MILP) capace di posizionare in maniera ottimale il minor numero possibile di droni per raggiungere tutti i client e soddisfare le loro richieste di traffico. Di particolare rilievo è il fatto di aver integrato nel modello una rappresentazione delle problematiche derivanti dai fenomeni di interferenza, che hanno un notevole impatto sulle capacità dei canali wireless utilizzati. A tal fine, abbiamo proposto un modello di interferenza locale e statico per stimare l'impatto della vicinanza reciproca dei droni e/o degli utenti sulla qualità delle comunicazioni. \\
Il modello è stato implementato in C++, utilizzando le librerie di CPLEX, al fine di ottenere con soluzioni off-the-shelf la soluzione ottima del problema. Abbiamo effettuato dei test preliminari per verificare le performance e i limiti del modello, e di fronte all'impossibilità di risolvere istanze medio-grandi in tempi ragionevoli abbiamo sviluppato un algoritmo euristico basata sulla soluzione iterativa di versioni semplificate del modello, ottenute fissando il valore di un opportuno insieme di variabili. \\
Infine abbiamo confrontato le performance degli algoritmi euristici contro quelle di CPLEX su un dataset creato ad-hoc, a causa dell'assenza in letteratura di dataset sufficientemente compatibili con il nostro problema, dimostrando che la complessità intrinseca del problema richieda espressamente l'adozione di algoritmi euristici per istanze di dimensioni realistiche. \\
Possibili sviluppi futuri del lavoro di tesi riguardano la ricerca di formulazioni alternative più efficienti, in termini di riduzione del numero di vincoli e variabili, l'estensione del modello per ridurre il più possibile il set di assunzioni fatte, migliorando così il realismo delle istanze, il perfezionamento del modello di interferenza, per esempio introducendo la presenza di ostacoli naturali e non (vegetazione, montagne, edifici, etc.), e integrandolo in un sistema di validazione per verificare la bontà dei parametri selezionati, iterando fasi di ottimizzazione-validazione, l'introduzione di modelli di random-walk, per simulare il movimento dei client, e di pattern di traffico realistico, l'ampliamento del dataset tramite funzioni di distribuzione dei clients più realistiche, la differenziazione delle caratteristiche di client e droni (diverse capacità e raggi trasmissivi), e lo sviluppo di nuove euristiche, basate su ricerca locale o algoritmi genetici, che risolvano il problema con approcci differenti (per esempio dal punto di vista geometrico dell'intersezione di cerchi, oppure della topologia dei grafi di rete).    
