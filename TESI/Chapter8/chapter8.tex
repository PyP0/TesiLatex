 \chapter{Conclusioni} \label{chap:conclusioni}

% **************************** Define Graphics Path **************************
\ifpdf
    \graphicspath{{Chapter8/Figs/Raster/}{Chapter8/Figs/PDF/}{Chapter8/Figs/}}
\else
    \graphicspath{{Chapter8/Figs/Vector/}{Chapter8/Figs/}}
\fi

\section{Conclusioni e sviluppi futuri}
L'impiego dei droni come strumenti di supporto per scopi civili è in costante crescita, e vengono studiati sempre nuovi metodi di impiego, spinti anche dalle concrete possibilità di business emergenti. \\
L'ambito sperimentale su cui si è concentrata questa tesi è quello di utilizzarli per il deployment rapido e autonomo di una rete di comunicazione di emergenza, da utilizzare in seguito del verificarsi di disastri naturali che hanno compromesso le reti di comunicazione tradizionali.
Per fare ciò i droni vengono equipaggiati con Access Points wireless e posizionati nel range dei client per creare una backbone FANET, consentendo loro di connettersi e comunicare. \\
In questa tesi abbiamo affrontato lo scenario appena descritto sviluppando un modello MILP che sia capace, sotto determinate ipotesi derivate dal vincolo di linearità, di posizionare in maniera ottimale il minor numero possibile di droni per raggiungere tutti i client e soddisfare le loro richieste di traffico. Abbiamo inoltre proposto un modello di interferenza locale e statico per stimare l'impatto della vicinanza reciproca dei droni sulla qualità delle comunicazioni. \\
Abbiamo effettuato dei test preliminari per verificare le performance e i limiti del modello, e di fronte all'impossibilità di risolvere istanze medio-grandi in tempi ragionevoli abbiamo sviluppato una soluzione euristica basata sul rilassamento continuo del problema originale e l'assegnamento di variabili. \\
Infine abbiamo confrontato le performance degli algoritmi euristici contro quelle di CPLEX su un dataset creato ad-hoc, a causa dell'assenza in letteratura di dataset sufficientemente compatibili con il nostro problema, dimostrando che la complessità intrinseca del problema richieda espressamente l'adozione di algoritmi euristici. \\
Possibili sviluppi futuri del nostro lavoro riguardano la ricerca di formulazioni alternative più efficienti, in termini di riduzione del numero di vincoli e variabili, l'estensione del modello per ridurre il più possibile il set di assunzioni fatte, migliorando così il realismo delle istanze, il perfezionamento del modello di interferenza, per esempio introducendo la presenza di ostacoli naturali e non (vegetazione, montagne, edifici, etc.), l'introduzione di modelli di random-walk, per simulare il movimento dei client, e di pattern di traffico realistico, l'ampliamento del dataset tramite funzioni di distribuzione dei clients più realistiche,e lo sviluppo di nuove euristiche, basate su ricerca locale o algoritmi genetici, che risolvano il problema con approcci differenti (per esempio dal punto di vista geometrico dell'intersezione di cerchi).    
